
%%% Local Variables:
%%% mode: latex
%%% TeX-master: t
%%% End:

\chapter{总结}
\label{cha:conclusion}
\section{本文工作总结}
本文针对大规模高维数据的近似近邻查询问题,首先对比研究了两大类索引结构——基于树结构的索引和基于哈希的索引。在树结构的索引中,本文介绍了两种用于近似近邻查询的索引树,分别是随机化 KD-树和分层 K-Means 树。在哈希的索引方法中,本文着重介绍谱哈希、K-Means 量化和 ITQ 量化三种哈希方法。在高维空间中,传统的基于树结构的索引方法会使得树的层数随着维度增加而不断增加,空间占用较大。从查询时间上来看,单次查询时间不仅受索引树深度的影响,同时也和查询数据的维度有关。基于哈希的方法对原始空间的数据集进行编码压缩,这类方法不仅减少了索引结构的占用空间,同时也一定程度上可以加快了查询效率。

通过对乘积量化的哈希方法的深入研究,我们在 Spark 平台上实现了一套基于乘积量化哈希方法的近似近邻查询系统。系统中采用基于 RDD 构建的分布式矩阵 BlockMatrix 来存储数据,使用 RDD 的持久化和减少通信数据量来不断优化。最终,通过在 SIFT1M、GIST1M 两个数据集上的实验证明,该系统在保证查询准确率的同时,通过并行计算的方式可以大大提高查询的时间效率。同时,我们也在 CIFAR-10 数据集上进行图像检索的任务,从而可以看出本系统的实用性。
\section{未来工作展望}
本文已经实现了一套基于 Spark 的近似近邻查询系统,采用乘积量化的方式进行哈希编码,但并没有在 Spark 上实现一套高效的索引方法。目前已被广泛应用的倒排索引方法或是新提出的多重倒排索引方法\cite{BabenkoL12}都是可用的索引方法。此外,编码方法也还存在改进的空间,近年来提出的 Cartesian K-Means\cite{Norouzi13} 在乘积量化量化的基础上,引入了旋转矩阵,在切分子空间将数据旋转矫正来减小数据和聚类中心的距离使得编码更加准确,这种方法在检索准确率上比乘积量化更高。

本文所进行实验在 SIFT1M、GIST1M、CIFAR-10 三个数据集上进行,未来的工作中可以考虑在 SIFT1B\footnote{http://corpus-texmex.irisa.fr/} 数据集和更大规模的图像或文本数据集上进行实验。最终,可以考虑在一个大规模的真实数据集上构建一个以图搜图的搜索引擎。



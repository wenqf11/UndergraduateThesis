
%%% Local Variables:
%%% mode: latex
%%% TeX-master: t
%%% End:
\secretlevel{绝密} \secretyear{2015}

\ctitle{基于 Spark 的分布式近似近邻查询系统}
% 根据自己的情况选,不用这样复杂
\makeatletter
\ifthu@bachelor\relax\else
  \ifthu@doctor
    \cdegree{工学博士}
  \else
    \ifthu@master
      \cdegree{工学硕士}
    \fi
  \fi
\fi
\makeatother


\cdepartment[软件]{软件学院}
\cmajor{计算机软件}
\cauthor{文庆福}
\csupervisor{王建民教授}
% 如果没有副指导老师或者联合指导老师,把下面两行相应的删除即可。
%\cassosupervisor{陈文光教授}
%\ccosupervisor{某某某教授}
% 日期自动生成,如果你要自己写就改这个cdate
%\cdate{\CJKdigits{\the\year}年\CJKnumber{\the\month}月}

% 博士后部分
% \cfirstdiscipline{计算机科学与技术}
% \cseconddiscipline{系统结构}
% \postdoctordate{2009年7月——2011年7月}

% 这块比较复杂,需要分情况讨论:
% 1. 学术型硕士
%    \edegree:必须为Master of Arts或Master of Science(注意大小写)
%              “哲学、文学、历史学、法学、教育学、艺术学门类,公共管理学科
%               填写Master of Arts,其它填写Master of Science”
%    \emajor:“获得一级学科授权的学科填写一级学科名称,其它填写二级学科名称”
% 2. 专业型硕士
%    \edegree:“填写专业学位英文名称全称”
%    \emajor:“工程硕士填写工程领域,其它专业学位不填写此项”
% 3. 学术型博士
%    \edegree:Doctor of Philosophy(注意大小写)
%    \emajor:“获得一级学科授权的学科填写一级学科名称,其它填写二级学科名称”
% 4. 专业型博士
%    \edegree:“填写专业学位英文名称全称”
%    \emajor:不填写此项


% 这个日期也会自动生成,你要改么?
% \edate{December, 2005}

% 定义中英文摘要和关键字
\begin{cabstract}
  近似近邻查询是处理多媒体数据的一项基本而重要的技术,在计算机视觉、数据挖掘等前沿研究领域有着广泛的应用。随着数据指数式的增长,如何从大规模高维数据中进行尽
  可能快速、精确地查询成为备受关注的问题。

  本文中,我们通过对比研究了基于树结构的索引方法和基于哈希的索引方法,并介绍了其中具有代表性的几种方法。本文采用向量量化的哈希方法,在 Spark 平台上建立了一套近似近邻
  查询系统,在保证高检索准确率的情况下加速查询效率。最终介绍了在三个数据集上进行的实验,验证系统的正确性和可用性。
\end{cabstract}

\ckeywords{近似近邻查询, Spark, 乘积量化, 索引}

\begin{eabstract}
   Approximate nearest neighbor search (ANNS) is a basic and important technique
   in processing multimedia data, which is widely used in frontier research fields,
   such as Computer Vision and Data Mining. With with exponential growth of data, much
   attention has been paid to the question that how to search in large-scale
   high-dimensional data as fast and accurately as we can.

   In this paper, we study tree-based indexing method and hash-based indexing
   method comparatively, and introduce several representative method of them. We build
   an ANNS system on Spark using hashing method based on vector quantization, which can
   improve query efficiency with high accuracy of retrieval. Finally, we introduce
   experiments that performed on three dataset to validate correctness and practicability
   of the system.
\end{eabstract}

\ekeywords{Approximate Nearest Neighbor Search, Spark, Product Quantization, Indexing}


%%% Local Variables:
%%% mode: latex
%%% TeX-master: t
%%% End:
\secretlevel{绝密} \secretyear{2015}

\ctitle{基于 Spark 的分布式数据量化与索引系统}
% 根据自己的情况选,不用这样复杂
\makeatletter
\ifthu@bachelor\relax\else
  \ifthu@doctor
    \cdegree{工学博士}
  \else
    \ifthu@master
      \cdegree{工学硕士}
    \fi
  \fi
\fi
\makeatother


\cdepartment[软件]{软件学院}
\cmajor{计算机软件}
\cauthor{文庆福}
\csupervisor{王建民教授}
% 如果没有副指导老师或者联合指导老师,把下面两行相应的删除即可。
%\cassosupervisor{陈文光教授}
%\ccosupervisor{某某某教授}
% 日期自动生成,如果你要自己写就改这个cdate
%\cdate{\CJKdigits{\the\year}年\CJKnumber{\the\month}月}

% 博士后部分
% \cfirstdiscipline{计算机科学与技术}
% \cseconddiscipline{系统结构}
% \postdoctordate{2009年7月——2011年7月}

% 这块比较复杂,需要分情况讨论:
% 1. 学术型硕士
%    \edegree:必须为Master of Arts或Master of Science(注意大小写)
%              “哲学、文学、历史学、法学、教育学、艺术学门类,公共管理学科
%               填写Master of Arts,其它填写Master of Science”
%    \emajor:“获得一级学科授权的学科填写一级学科名称,其它填写二级学科名称”
% 2. 专业型硕士
%    \edegree:“填写专业学位英文名称全称”
%    \emajor:“工程硕士填写工程领域,其它专业学位不填写此项”
% 3. 学术型博士
%    \edegree:Doctor of Philosophy(注意大小写)
%    \emajor:“获得一级学科授权的学科填写一级学科名称,其它填写二级学科名称”
% 4. 专业型博士
%    \edegree:“填写专业学位英文名称全称”
%    \emajor:不填写此项


% 这个日期也会自动生成,你要改么?
% \edate{December, 2005}

% 定义中英文摘要和关键字
\begin{cabstract}
  高维数据的近邻查询是处理多媒体数据的一项重要的技术,在计算机视觉、数据挖掘等
  前沿研究领域有着丰富的应用。随着数据指数式的增长,如何从海量、高维数据中进行尽
  可能快速的近邻查询以及大规模的数据如何存储索引,一直以来都是备受研究者关注的问
  题。
  
  论文的摘要是对论文研究内容和成果的高度概括。摘要应对论文所研究的问题及其研究目
  的进行描述,对研究方法和过程进行简单介绍,对研究成果和所得结论进行概括。摘要应
  具有独立性和自明性,其内容应包含与论文全文同等量的主要信息。使读者即使不阅读全
  文,通过摘要就能了解论文的总体内容和主要成果。
  
\end{cabstract}

\ckeywords{近邻查询, Spark, 向量量化, 倒排索引}

\begin{eabstract}
   An abstract of a dissertation is a summary and extraction of research work
   and contributions. Included in an abstract should be description of research
   topic and research objective, brief introduction to methodology and research
   process, and summarization of conclusion and contributions of the
   research. An abstract should be characterized by independence and clarity and
   carry identical information with the dissertation. It should be such that the
   general idea and major contributions of the dissertation are conveyed without
   reading the dissertation.

   An abstract should be concise and to the point. It is a misunderstanding to
   make an abstract an outline of the dissertation and words ``the first
   chapter'', ``the second chapter'' and the like should be avoided in the
   abstract.

   Key words are terms used in a dissertation for indexing, reflecting core
   information of the dissertation. An abstract may contain a maximum of 5 key
   words, with semi-colons used in between to separate one another.
\end{eabstract}

\ekeywords{\TeX, \LaTeX, CJK, template, thesis}

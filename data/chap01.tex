
%%% Local Variables:
%%% mode: latex
%%% TeX-master: t
%%% End:

\chapter{引言}
\label{cha:introdction}

\section{研究背景}
在今天这个数据爆炸的时代,文本、图像、视频以及音频等非结构化数据呈现出指数级的增长。如何快速、准确地从这个海量的互联网数据库中获取我们想要的信息,是我们不得不面对的一个问题。Google\footnote{https://www.google.com}、Bing\footnote{http://cn.bing.com}、Baidu\footnote{https://www.baidu.com} 等提供的文本、图像等搜索服务为我们获取信息带来了极大的便利。而在这些搜索引擎背后的都需要用到的一项技术——近似近邻查询(Approximate Nearest Neighbor Search)。在大规模数据的应用场景下,精确的近似查询需要耗费时间太长,不具有实际应用价值。近似近邻查询可以大幅度缩短查询时间,同时保证查询结果与精确查询结果近似,因此更具有实用性。除了信息检索以外,近似近邻查询技术被广泛应用于模式识别、机器学习、数据挖掘等领域。

近年来,近似近邻查询技术一直都是研究热点之一,但是该技术面临的挑战却没有改变。

一方面,随着互联网上的数据越来越多,需要处理的数据量也越来越大。然而,传统的索引结构一般都是基于小规模数据而设计的单机结构。大规模的数据一般无法做到单机存储,更不用说加载到内存当中进行索引了。这些数据往往存储于分布式系统当中,同时也需要一种分布式的索引结构来支持查询。海量的数据不仅给存储数据带来了压力,同时也给实时数据查询带来了巨大挑战。

另一方面,在非结构化数据处理过程时,往往都会针对非结构化数据提取特征进行处理。为了更好地表示数据,一般特征数据的维度越高,特征表示的准确性也越高。例如,在图像数据处理过程中,我们常常可能用到的 SIFT、SURF 特征都是 128 维的,GIST 特征有 960 维,而 BOVW(Bag of Visual Words)的维度更是高达成千上万维。在如此高维度的数据如何进行快速、高效的检索是一个必须解决的问题。

面对大数据处理的现状,Hadoop、Storm、Spark 等一系列的大数据并行计算框架相继涌现出来。本文中,我们基于 Spark 平台,在其上构建一套近似近邻查询系统来实现快速的高维数据检索。我们希望利用 Spark 并行计算的特点,使我们构建的系统能够应对大规模数据进行高效的近似近邻查询。

对于近似近邻查询问题,在不考虑时间效率的情况下,这一问题可以直接通过一种暴力搜索的方式来解决。比如,我们可以直接计算查询数据 $q$ 与数据集合 $S$ 中每一条数据的距离,最终根据距离大小,选取出距离最近的前 $n$ 个数据。但在现实中,由于数据集合 $S$ 的规模非常大,这种朴素的查询方法的计算时间太长而无法实用。但是,如果从规模比较大的数据集合 $S$ 上删选出一个非常小待选集合 $S'$,之后再在集合 $S'$ 上进行朴素的暴力搜索选取出前 $n$ 个近邻数据,这时的暴力搜索的时间效率是可以接受的,整个查询过程的时间效率和准确率就取决于删除出待选集合 $S'$ 的过程。待选集合 $S'$ 大小与查询准确率有着密切关系,一般来说,集合 $S'$ 越大,查询准确率越高,但是最终的暴力搜索阶段的时间会变长;集合 $S'$ 越小,则查询效率越低,暴力搜索阶段的时间越短。因此,如何快速地选取出一个大小合适、相关性高的待选集合 $S'$ 就成了近似近邻查询问题的关键。

\section{主要工作}
本文的主要的关注点是在高维空间中的近似近邻查询问题,通过研究对比现有的近似近邻查询方法,了解几种不同方法各自的特点。在 Spark 框架下实现了一套的近似近邻查询系统。最终,通过实验来验证准确性以及测试时间效率。

总体而言,本文的主要工作包括三个方面:
\begin{itemize}
\item 算法研究:通过阅读文献,调研现有的近似近邻查询方法,并对现有方法进行比较和分析。
\item 系统设计与实现:基于算法研究的基础,在 Spark 平台上设计并实现了一套基于乘积量化的近似近邻查询系统。该系统不仅可以对数据进行编码压缩以减少空间占用,而且利用并行计算可以大幅度加快查询速度。
\item 实验验证:通过在不同的数据集上进行实验,验证系统的正确性与可靠性。
\end{itemize}

\section{论文组织结构}
本文首先介绍了研究背景和本文的主要工作。在接下来的章节中,第二章中对现有的索引方法的进行了综述并对其中代表性方法进行介绍;第三章首先介绍了 Spark 并行计算框架中弹性分布式数据集与分布式机器学习库 MLlib ,之后介绍了乘积量化的近似近邻查询方法以及我们在 Spark 平台上设计与实现;第四章介绍了我们实现的系统在三个不同数据集上进行的实验,对实验结果进行了分析;最终,第五章节对本文工作进行了总结,并指出了本文工作中还需要改进的地方。
